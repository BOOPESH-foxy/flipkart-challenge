\documentclass[a4paper]{article}

\usepackage[utf8]{inputenc}
\usepackage[T1]{fontenc}
\usepackage{textcomp}
\usepackage[dutch]{babel}
\usepackage{amsmath, amssymb}
\usepackage{code}
\usepackage{pythonhighlight}

% figure support
\usepackage{import}
\usepackage{xifthen}
\pdfminorversion=7
\usepackage{pdfpages}
\usepackage{transparent}
\usepackage{graphicx}
\pdfsuppresswarningpagegroup=1
\graphicspath{{./img/}}

\begin{document}
\bibliographystyle{article}
\section{Kinematics for omniwheel }
Kinematics of four wheel omni wheel robot

\begin{equation}\label{kinematics}   
    \begin{bmatrix}
		\omega_1 \\
		\omega_2 \\
		\omega_3 \\
		\omega_3 \\
	\end{bmatrix}
	=\frac{1}{R}
	\begin{bmatrix}
		\sin(\alpha_1) & \cos(\alpha_1) & R \\
		\sin(\alpha_2) & \cos(\alpha_2) & R \\
		\sin(\alpha_3) & \cos(\alpha_2) & R \\
		\sin(\alpha_4) & \cos(\alpha_2) & R \\
	\end{bmatrix}
	\begin{bmatrix}
		V_x \\
		V_y \\
		\theta
	\end{bmatrix}
\end{equation}
\subsection{Bot dimension}
\begin{itemize}
	\item R (wheel radius) = 29mm
	\item $\alpha_1 = 0 $
	\item $\alpha_2 = \frac{\pi}{2} $
	\item $\alpha_3 = \pi $
	\item $\alpha_2 = \frac{3\pi}{4} $
\end{itemize}
On equating \ref{kinematics} we get
\begin{equation}
	\begin{bmatrix}
		\omega_1 \\
		\omega_2 \\
		\omega_3 \\
		\omega_3 \\
	\end{bmatrix}
	= 1/R \begin{bmatrix}
		V_y + R \theta  \\
		V_x + R \theta  \\
		-V_y + R \theta \\
		-V_x + R \theta \\
	\end{bmatrix}
\end{equation}
\section{PID tuning }
\begin{itemize}
        \item Increase the proportional constant by factor of 10 until you get closer oscilation to the set state
        \item Increase the derivative constant by factor of 10 until the oscilation dampens
        \item increase $p$ and  $d$ slowly and proportionaly until the procces state gets closer to set value
        \item Increase $i$ slowly to get required state 
\end{itemize}
\bibliography{notes}
\end{document}
